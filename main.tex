% !TEX TS-program = xelatex
% !BIB program = bibtex
% !TEX encoding = UTF-8 Unicode
% !Mode:: "TeX:UTF-8"
 \documentclass[bachelor, nocolorlinks, printoneside]{seuthesis} % 本科
% \documentclass[master]{seuthesis} % 硕士
% \documentclass[doctor]{seuthesis} % 博士
% \documentclass[engineering]{seuthesis} % 工程硕士
\usepackage{CJK,CJKnumb}
\usepackage{amsmath}
\usepackage{amsfonts} 
\usepackage{bm} 
\usepackage{algorithm}
\usepackage{algorithmicx}
\usepackage{algpseudocode}
\usepackage{subfigure}




\floatname{algorithm}{算法}
\renewcommand{\algorithmicrequire}{\textbf{输入:}}
\renewcommand{\algorithmicensure}{\textbf{输出:}}
 % 这里是导言区

\begin{document}
\categorynumber{000} % 分类采用《中国图书资料分类法》
\UDC{000}            %《国际十进分类法UDC》的类号
\secretlevel{公开}    %学位论文密级分为"公开"、"内部"、"秘密"和"机密"四种
\studentid{04216747}   %学号要完整,前面的零不能省略。

\title{基于深度学习的三维重建和点云生成}{}{SEU Thesis \LaTeX Template}{subtitle}
\author{周烨凡}{Yefan Zhou}
\advisor{杨绿溪}{教授}{No Name}{Prof.}
\department{信息科学与工程学院}{Radio}
\major[12em]{信息工程}
\duration{2020年1月-2020年5月}

%\advisor{无名氏}{教授}{No Name}{Prof.}



%\maketitle

\begin{abstract}{三维重建,点云,深度学习,数据挖掘}

    单视角三维物体形状重建(Single-view 3D Reconstruction)是三维视觉领域一直以来的核心问题之一。由于拟合非线性方程和学习模式的有效性,深度神经网络被期望在重建三维非线性形状的任务中表现出色。本课题探究了深度神经网络在三维重建任务中的有效性和内在机制,主要的工作和创新如下:
        
        \begin{enumerate}
            \item[1.] 针对基于自编码器结构的深度神经网络多次训练结果差异较大的问题,提出采用多任务训练对编码器进行初始化,以及引入残差层增加解码器网络深度这两种方法优化网络架构。在ShapeNet公开数据集上进行测试,取得了优于基于检索的非学习方法Oracle Nearest Neighbor的结果。
            \item[2.] 为了衡量三维数据集的聚类程度,提出了基于affinity propagation和silhouette score的度量标准,该方法得到的量化结果与数据集降维后的可视化结果吻合。
            \item[3.] 提出了影响三维重建任务中深度神经网络内在机制的主要因素是数据集的本质特征:当训练集中的三维形状集有相比于图片集更高的聚类程度的结构时,在这个数据集上训练的深度神经网络更有可能会执行识别机制而非重建机制,并定性和定量的证明了这一强相关性。
        \end{enumerate}
    最后,本文分析了上述结论对三维重建任务中的数据收集和神经网络训练的指导意义。
    \end{abstract}

\begin{englishabstract}{3D Reconstruction, Point Cloud, Deep Learning, Data mining}
    Single-view 3D reconstruction is one of the elementary tasks in the field of 3D vision. Due to the effectiveness in learning and approximating non-linear function, deep neural networks are expected to perform well on the task of reconstructing 3D nonlinear shapes. This paper investigates the effectiveness and internal mechanism of deep neural network in 3D reconstruction task. The main contribution and innovation are summarized as follows:
    \begin{enumerate}
        \item[1.] For the problem that the variance of multiple training results of deep neural network based on autoencoder structure is too large, it proposes to leverage multi-task training to initiate the encoder and introduce residual layer to deepen the decoder in order to optimize network architecture. These methods are trained and tested on ShapeNet public dataset and they outperform the non-learning method based on retrieve, namely Oracle Nearest Neighbor. 
        \item[2.] Define novel way to measure clustering coefficient of 3D reconstruction dataset based on affinity propagation and silhouette score, the quantitative results of this metric is corresponding to qualitative results of visualization of dataset in low dimension.
        \item[3.] It claims that the bias of internal mechanism of network is mainly affected by the intrinsic properties of dataset: when the training set of 3D shapes has a more clustered structure than images, the deep neural networks trained on this dataset become more likely to perform recognition than reconstruction, and it proves the strong correlation between the deep neural networks and dataset property both qualitatively and quantitatively.
    \end{enumerate}
    Finally, this paper analyzes the significance of the above conclusions for data collection and neural network training.
\end{englishabstract}

\tableofcontents

% \begin{terminology}
% \begin{table}[h]
% \renewcommand\arraystretch{1.5}
% %\Large
% \begin{tabular}{>{\LARGE}m{0.2\textwidth} <{\centering}m{0.7\textwidth}}
% a & 如同汉字起源于象形,拉丁字母表中的每个字母一开始都是描摹某种动物或物体形状的图画\\

% b&和A一样,字母B也可以追溯到古代腓尼基。在腓尼基字母表中B叫beth,代表房屋,在希伯来语中B也叫beth,也含房屋之意。\\

% c& 字母C在腓尼基人的文字中叫gimel,代表骆驼。它在字母表中的排列顺序和希腊字母Γ(gamma)相同,实际上其字形是从后者演变而来的。C在罗马数字中表示100。\\

% d&D在古时是描摹拱门或门的形状而成的象形符号,在古代腓尼基语和希伯来语中叫做daleth,是“门”的意思,相当于希腊字母Δ(delta)。\\

% \end{tabular}
% %\caption{my table}
% \end{table}
% \end{terminology}


\begin{Main} % 开始正文

\chapter{前言}
%\emph{在泛函分析中,卷积、旋积或摺积(英语:Convolution)是通过两个函数f 和g 生成第三个函数的一种数学算子,表征函数f 与g经过翻转和平移的重叠部分的面积。}

\section{数学公式}
\subsection{简单的数学公式}
\textbf{卷积}(\textbf{convolution})在图像分析的线性方法中是一种重要的运算。卷积是一个积分,反映一个函数$f(t)$在另一个函数上$h(t)$移动时所重叠的量。函数$f$和$h$在有限域$[0,t]$上的$1D$卷积$f*h$由下式给出:
 $$(f*h)(t) \equiv \int_0^t {f(\tau )h(t - \tau )d\tau } $$

\subsection{带自动编号的公式}
这里可以限定在$[0,t]$区间,原因是我们假设负坐标部分的值是0。为了准确起见,我们还可以将卷积积分的上限扩展为$( - \infty ,\infty )$:
\begin{equation}(f*h)(t) \equiv \int_{ - \infty }^\infty  {f(\tau )h(t - \tau )d\tau }  = \int_{ - \infty }^\infty  {f(t - \tau )h(\tau )d\tau }
  \end{equation} 

\subsection{带等号对齐的公式}
卷积可以推广到更高维。令$2D$函数$f$和$h$的卷积$g$记为$f*h$,则有:
\begin{equation}
\begin{aligned}
(f*h)(x,y) &= \int_{ - \infty }^\infty  {\int_{ - \infty }^\infty  {f(a,b)h(x - a,y - b)} } dadb\\
 &= \int_{ - \infty }^\infty  {\int_{ - \infty }^\infty  {f(x - a,y - b)h(a,b)} } dadb\\
\end{aligned}
\end{equation}

\section{伪代码}
在写论文的时候我们通常要写伪代码,伪代码里面有时甚至还要包含数学公式(如根号一类的)。伪代码会自动找一个比较好的位置插入图片。

\begin{algorithm}
    \caption{用归并排序求逆序数}
    \begin{algorithmic}[1] %每行显示行号
        \Require $Array$数组,$n$数组大小
        \Ensure 逆序数
        \Function {MergerSort}{$Array, left, right$}
            \State $result \gets 0$
            \If {$left < right$}
                \State $middle \gets (left + right) / 2$
                \State $result \gets result +$ \Call{MergerSort}{$Array, left, middle$}
                \State $result \gets result +$ \Call{MergerSort}{$Array, middle, right$}
                \State $result \gets result +$ \Call{Merger}{$Array,left,middle,right$}
            \EndIf
            \State \Return{$result$}
        \EndFunction
        \State
        \Function{Merger}{$Array, left, middle, right$}
            \State $i\gets left$
            \State $j\gets middle$
            \State $k\gets 0$
            \State $result \gets 0$
            \While{$i<middle$ \textbf{and} $j<right$}
                \If{$Array[i]<Array[j]$}
                    \State $B[k++]\gets Array[i++]$
                \Else
                    \State $B[k++] \gets Array[j++]$
                    \State $result \gets result + (middle - i)$
                \EndIf
            \EndWhile
            \While{$i<middle$}
                \State $B[k++] \gets Array[i++]$
            \EndWhile
            \While{$j<right$}
                \State $B[k++] \gets Array[j++]$
            \EndWhile
            \For{$i = 0 \to k-1$}
                \State $Array[left + i] \gets B[i]$
            \EndFor
            \State \Return{$result$}
        \EndFunction
    \end{algorithmic}
\end{algorithm}

\section{插入图片}
在使用该命令的时候,图片会自动找一个他觉得比较好的位置插入图片,我们就不需要担心前面改了文字之后后面的格式乱掉。
\begin{figure}[htbp!]
\centering \includegraphics[width=0.9\textwidth]{img/test.jpg} \caption{图片的一个简单应用场景}
\end{figure}

\begin{figure}
\centering
\subfigure[the first subfigure]{
\includegraphics[width=0.4\textwidth]{img/test.jpg} 
}
\subfigure[the second subfigure]{
\includegraphics[width=0.4\textwidth]{img/test.jpg} 
}
\caption{子图应用场景}
\end{figure}

\section{引用论文}
使得论文符合要求\cite{Yao:2015ix}\cite{seucover}。

\chapter{研究内容}
\section{本章小结}

\end{Main} % 结束正文

% 参考文献
\bibliography{seuthesis}

%附录
\begin{Appendix}{}
    \chapter{{\LaTeX}实验}
    技术实验结果在这里写
    \chapter{MATLAB实验}
    技术实验结果在这里写
\end{Appendix}

\begin{Acknowledgement}{}
    这次的毕业论文设计总结是在我的指导老师xxx老师亲切关怀和悉心指导下完成的。从毕业设计选题到设计完成,x老师给予了我耐心指导与细心关怀,有了莫老师耐心指导与细心关怀我才不会在设计的过程中迷失方向,失去前进动力。x老师有严肃的科学态度,严谨的治学精神和精益求精的工作作风,这些都是我所需要学习的,感谢x老师给予了我这样一个学习机会,谢谢!

    感谢与我并肩作战的舍友与同学们,感谢关心我支持我的朋友们,感谢学校领导、老师们,感谢你们给予我的帮助与关怀;感谢肇庆学院,特别感谢计算机科学与软件学院四年来为我提供的良好学习环境,谢谢!
\end{Acknowledgement}

\newpage
\printindex % 索引

%\begin{thebibliography}{99}


%\bibliographystyle{ieee}
%\bibliography{seuthesis}


\end{document}
